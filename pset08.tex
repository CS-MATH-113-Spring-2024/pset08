\documentclass[a4paper]{exam}


\usepackage{amsfonts,amsmath,amsthm}
\usepackage[a4paper]{geometry}
\usepackage{hyperref}
\usepackage{xcolor}

\theoremstyle{definition}
\newtheorem{definition}{Definition}

\newcommand\Z{\ensuremath{\mathbb{Z}}}
\newcommand\R{\ensuremath{\mathbb{R}}}

\title{Problem Set 08: Sets And Functions}
\author{CS/MATH 113 Discrete Mathematics}
\date{Spring 2024}

\boxedpoints

\printanswers

\begin{document}
\maketitle

\begin{questions}
\question 
  Prove: $\mathcal{P}(A) \subseteq \mathcal{P}(B)$ if and only if $A \subseteq B$.

  \begin{solution}
    % Enter your solution here.
  \end{solution}


\question Show that if \(A \subseteq C\) and \(B \subseteq D\), then \(A \times B \subseteq C \times D\).

  \begin{solution}
    % Enter your solution here.
  \end{solution}
  
\question Prove or disprove the following statements for all sets $A, B,$ and $C$:
  \begin{parts}
  \part $ A \times (B \cup C) = (A \times B) \cup (A \times C) $
    \begin{solution}
      % Enter your solution here.
    \end{solution}

  \part $A \times (B \cap C) = (A \times B) \cap (A \times C) $
    \begin{solution}
      % Enter your solution here.
    \end{solution}
  \end{parts}
  
\question If $A, B,$ and $C$ are sets such that $A \subseteq B$ and $B \subseteq C$, show that $A \subseteq C$.

  \begin{solution}
    % Enter your solution here.
  \end{solution}

\question Show that if $A$ and $B$ are finite sets, then  \( A \cup B \) is a finite set.

  \begin{solution}
    % Enter your solution here.
  \end{solution}
  
\question
  \begin{parts}
  \part Prove that the function $f: \Z\to\Z$, defined as \(f(n) = n^2\) is neither injective nor surjective.
    \begin{solution}
      % Enter your solution here.
    \end{solution}

  \part Can you make this function injective without doing any changes to the original function? (Hint: try redefining its domain and/or co-domain.)
    \begin{solution}
      % Enter your solution here.
    \end{solution}
  \end{parts}

  
\question Prove that the function $f: \Z\to\Z$ defined as \(f(n) = n+1\), is invertible.

  \begin{solution}
    % Enter your solution here.
  \end{solution}
  
\question Give an explicit formula for a function from \Z to $\Z^+$ that is:
  \begin{parts}
  \part one-to-one, but not onto,
    \begin{solution}
      % Enter your solution here.
    \end{solution}

  \part onto, but not one-to-one,
    \begin{solution}
      % Enter your solution here.
    \end{solution}

  \part one-to-one and onto, and
    \begin{solution}
      % Enter your solution here.
    \end{solution}

  \part neither one-to-one nor onto.
    \begin{solution}
      % Enter your solution here.
    \end{solution}
  \end{parts}

\question A function \(f: I \rightarrow \R\) is strictly increasing on an interval \(I\) if for all \(x_1, x_2 \in I\) with \(x_1 < x_2\), it holds that \(f(x_1) < f(x_2)\). Here $I$ represents the interval on which the function is strictly increasing.
  
  Given this definition, prove that a strictly increasing function is always injective over the interval, I.

  \begin{solution}
    % Enter your solution here.
  \end{solution}
\end{questions}
\end{document}
%%% Local Variables:
%%% mode: latex
%%% TeX-master: t
%%% End: