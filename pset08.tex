\documentclass[a4paper]{exam}

\usepackage{amsfonts,amsmath,amsthm}
\usepackage[a4paper]{geometry}
\usepackage{hyperref}
\usepackage{xcolor}

\theoremstyle{definition}
\newtheorem{definition}{Definition}

\newcommand\Z{\ensuremath{\mathbb{Z}}}
\newcommand\R{\ensuremath{\mathbb{R}}}
\newcommand\union{\cup}
\newcommand\interx{\cap}

\usepackage{draftwatermark}
\SetWatermarkText{Sample Solution}
\SetWatermarkScale{3}
\printanswers

\title{Problem Set 08: Sets And Functions}
\author{CS/MATH 113 Discrete Mathematics}
\date{Spring 2024}

\boxedpoints

\printanswers

\begin{document}
\maketitle

\begin{questions}
\question 
  Prove: $\mathcal{P}(A) \subseteq \mathcal{P}(B)$ if and only if $A \subseteq B$.

  \begin{solution}
    The statement to prove is a biconditional. Proving it requires to prove two conditionals.

    First, we show that $\mathcal{P}(A) \subseteq \mathcal{P}(B)$ if $A \subseteq B$. We prove directly that an element of $\mathcal{P}(A)$ is an element of $\mathcal{P}(B)$ if $A$ is a subset of $B$.
    
    \begin{proof}$A \subseteq B \implies \mathcal{P}(A) \subseteq \mathcal{P}(B)$.\\
      Consider $A \subseteq B$.\hfill (premise)\\
      Consider an element, $C$, in $\mathcal{P}(A)$. That is, $C \in \mathcal{P}(A)$.\hfill(standard notation)\\
      $\implies$ $C\subseteq A$.\hfill(definition of powerset)\\
      $\implies$ every element of $C$ is a member of $A$.\hfill(definition of subset)\\
      $\implies$ every element of $C$ is a member of $B$.\hfill($A\subseteq B$, definition of subset)\\
      $\implies$ $C\subseteq B$.\hfill(definition of subset)\\
      $\implies$ $C\in \mathcal{P}(B)$.\hfill(definition of powerset)
    \end{proof}

    Next, we prove the remaining conditional, $\mathcal{P}(A) \subseteq \mathcal{P}(B)$ only if $A \subseteq B$. We show directly that, given the premise, $A$ must be a subset of $B$.
    
    \begin{proof}$\mathcal{P}(A) \subseteq \mathcal{P}(B)\implies A \subseteq B$.\\
      $A \in \mathcal{P}(A)$.\hfill (definition of powerset)\\
      $\implies$ $A \in \mathcal{P}(B)$.\hfill ($\mathcal{P}(A) \subseteq \mathcal{P}(B)$, definition of subset)\\
      $\implies$ $A \subseteq B$.\hfill (definition of powerset)
    \end{proof}
  \end{solution}

\question Show that if \(A \subseteq C\) and \(B \subseteq D\), then \(A \times B \subseteq C \times D\).

  \begin{solution}
    We attempt a direct proof. We show that under the given premise, an element of \(A \times B\) is also present in $C \times D$.
    \begin{proof}\(((A \subseteq C)\land (B \subseteq D))\implies (A \times B \subseteq C \times D)\)\\
      We assume that $A$ is a subset of $C$ and that $B$ is a subset of $D$.\hfill(premise)\\
      Consider an element $(a,b)$ of $A\times B$. Then $a\in A$ and $b\in B$.\hfill(definition of set product)\\
      $\implies$ $a$ is an element of $C$, and $b$ is an element of $D$.\hfill(premise, definition of subset)\\
      $\implies$ $(a,b)$ is an element of $C\times D$.\hfill(definition of set product)
    \end{proof}
  \end{solution}
  
\question Prove or disprove the following statements for all sets $A, B,$ and $C$:
  \begin{parts}
  \part $ A \times (B \union C) = (A \times B) \union (A \times C) $
    \begin{solution}
      We prove that the set on each side of the equality is a subset of the other. That is, two statements needs to be proved. We provide a direct proof of each. That is, we show that an element of one is also present in the other.

      \begin{proof}$ A \times (B \union C) \subseteq (A \times B) \union (A \times C) $\\
        Consider $(a,d)\in A\times (B\union C)$.\\
        $\implies$ $a\in A$ and $d\in B$ or $d\in C$.\hfill(definitions of set product and union)\\
        If $d\in B$, then $(a,d)\in A\times B$.\hfill(definition of set product)\\
        $\implies$ $(a,d)\in (A\times B)\union(A\times C)$.\hfill(definition of union)\\
        Otherwise, if $d\in C$, then $(a,d)\in A\times C$.\hfill(definition of set product)\\
        $\implies$ $(a,d)\in (A\times B)\union(A\times C)$.\hfill(definition of union)
      \end{proof}

      \begin{proof}$ (A \times B) \union (A \times C)\subseteq A \times (B \union C) $\\
        Consider $(a,d)\in (A \times B) \union (A \times C)$.\\
        $\implies$ $(a,d)\in A\times B$ or $(a,d)\in A\times C$.\hfill(definition of union)\\
        If $(a,d)\in A\times B$ then $(a,d)\in A\times (B\union C)$.\hfill(definitions of union, product)\\
        Otherwise, if $(a,d)\in A\times C$ then $(a,d)\in A\times (B\union C)$.\hfill(as above)
      \end{proof}
    \end{solution}

  \part $A \times (B \interx C) = (A \times B) \interx (A \times C) $
    \begin{solution}
      We prove that the set on each side of the equality is a subset of the other. That is, two statements needs to be proved. We provide a direct proof of each. That is, we show that an element of one is also present in the other.

      \begin{proof}$(A \times (B \interx C))\subseteq ((A \times B) \interx (A \times C))$\\
        Consider $(a,d)\in A\times (B\interx C)$.\\
        $\implies$ $d\in B$ and $d\in C$.\hfill(definitions of intersection, set product)\\
        Because $d\in B$, $(a,d)$ is present in $A\times B$.\hfill(definition of set product)\\
        And because $d\in C$, $(a,d)$ is present in $A\times C$.\hfill(as above)\\
        $\implies$ $(a,d)\in (A\times B) \interx (A\times C)$\hfill(definition of intersection)
      \end{proof}

      \begin{proof}$ ((A \times B) \interx (A \times C)) \subseteq (A \times (B \interx C))$\\
        Consider $(a,d)\in (A \times B) \interx (A \times C)$.\\
        $\implies$ $(a,d)\in A\times B$ and $(a,d)\in A\times C$.\hfill(definition of intersection)\\
        Because $(a,d)\in A\times B$ then $d\in B$.\hfill(definition of set product)\\
        And because $(a,d)\in A\times C$ then $d\in C$.\hfill(as above)\\
        $\implies$ $d\in B$ and $d\in C$.\hfill(conjunction)\\
        $\implies$ $d\in (B\interx C)$.\hfill(definition of intersection)\\
        $\implies$ $(a,d)\in A\times (B\interx C)$.\hfill(definition of set product)
      \end{proof}
    \end{solution}
  \end{parts}
  
\question If $A, B,$ and $C$ are sets such that $A \subseteq B$ and $B \subseteq C$, show that $A \subseteq C$.

  \begin{solution}
    We prove directly that, given the premises, an element of $A$ is an element of $C$.

    \begin{proof}
      Consider $a\in A$.\\
      $\implies$ $a\in B$.\hfill(premise, definition of subset)\\
      $\implies$ $a\in C$.\hfill(premise, definition of subset)
    \end{proof}
  \end{solution}

\question Show that if $A$ and $B$ are finite sets, then  \( A \union B \) is a finite set.

  \begin{solution}
    % Enter your solution here.
  \end{solution}
  
\question
  \begin{parts}
  \part Prove that the function $f: \Z\to\Z$, defined as \(f(n) = n^2\) is neither injective nor surjective.
    \begin{solution}
      % Enter your solution here.
    \end{solution}

  \part Can you make this function injective without doing any changes to the original function? (Hint: try redefining its domain and/or co-domain.)
    \begin{solution}
      % Enter your solution here.
    \end{solution}
  \end{parts}

  
\question Prove that the inverse of the function $f: \Z\to\Z$ defined as \(f(n) = n+1\), is also a function.

  \begin{solution}
    % Enter your solution here.
  \end{solution}
  
\question Give an explicit formula for a function from \Z to $\Z^+$ that is:
  \begin{parts}
  \part one-to-one, but not onto,
    \begin{solution}
      % Enter your solution here.
    \end{solution}

  \part onto, but not one-to-one,
    \begin{solution}
      % Enter your solution here.
    \end{solution}

  \part one-to-one and onto, and
    \begin{solution}
      % Enter your solution here.
    \end{solution}

  \part neither one-to-one nor onto.
    \begin{solution}
      % Enter your solution here.
    \end{solution}
  \end{parts}

\question A function \(f: I \rightarrow \R\) is strictly increasing on an interval \(I\) if for all \(x_1, x_2 \in I\) with \(x_1 < x_2\), it holds that \(f(x_1) < f(x_2)\). Here $I$ represents the interval on which the function is strictly increasing.
  
  Given this definition, prove that a strictly increasing function is always injective.

  \begin{solution}
    % Enter your solution here.
  \end{solution}
\end{questions}
\end{document}
%%% Local Variables:
%%% mode: latex
%%% TeX-master: t
%%% End: